\documentclass{beamer}
\usetheme{Marburg}
\usepackage{amsmath, amsthm, amsfonts, amssymb, epsfig, subfigure, boxedminipage}
\usepackage{epstopdf}
\usepackage[english]{babel}
\usepackage[latin1]{inputenc}
\usepackage{movie15}
\usepackage{graphicx}
\usepackage{hyperref}
\newcommand{\dif}{\textrm{d}}
\def\normv{ {\bf e}_{\eta}}
\def\ter{ \text{t}}
\def\edge{ \text{e}}
\def\ux{ {\bf e}_{x}}
\def\uy{ {\bf e}_{y}}

\usepackage{etoolbox}
\makeatletter
\patchcmd{\insertverticalnavigation}%
{\ifx\beamer@nav@css\beamer@hidetext{\usebeamertemplate{section in sidebar}}\else{\usebeamertemplate{section in sidebar shaded}}\fi}
{{\usebeamertemplate{section in sidebar}}}{}{}
\makeatother

%\makeatletter
%\patchcmd{\insertverticalnavigation}%
%{\ifx\beamer@nav@css\beamer@hidetext{\usebeamertemplate{subsection in sidebar}}\else{\usebeamertemplate{subsection in sidebar shaded}}\fi}
%{{\usebeamertemplate{subsection in sidebar}}}{}{}
%\makeatother

\setbeamercovered{transparent}
\title{Tips, Tools, and Resources \\for\\ Teaching an Active-Learning (motivated)\\Differential Equations Course}
\author{Dr. Karl Schmitt, Valparaiso University}
\begin{document}

\section{Introduction}
\frame{
\begin{center}
\titlepage
Joint Mathematics Meeting 2016\\

%Febuary $1^{st}$, 2013
\end{center}
}

\frame{
\frametitle{No Need to Copy}
Download slides at:\\
\url{https://github.com/Pelonza/JMM16-ODE-Talk}
}

\frame{
\frametitle{Background}
\vspace{-.25cm}
Course\\
\begin{itemize}
	\item Mixture of Engineering (Mech, ECE), Mathematics, and Meteorology students
	\item Typical class size 15-30
	\item Students have limited programming experience by this course
\end{itemize}
Teaching Goals
\begin{itemize}
	\item Be able to use class-time as exploratory/active learning
	\item Integrate significant computational explorations (6-10 for the semester)
	\item Provide sufficient in-depth applications to motivate learning DEs
\end{itemize}
}

%Outline
%Textbooks -- Woodruff's IBL, Jiri Lebl + IODE,  

\section{Resources}
\frame{
\frametitle{Textbook Choices}
\begin{itemize}
	\item \textit{Differential Equations Laboratory Workbook},\\
	Borrelli, Coleman, and Boyce\\
	Available on CODEE:\\ \tiny{\url{www.codee.org/library/projects/differential-equations-laboratory-workbook-1}}\normalsize
	\begin{itemize}
		\item Lots(!!) of experiments
		\item Originally from 1992, so specific software references very out-of-date
		\item If planned well and/or updated can serve well as labs or homework activities.
	\end{itemize}
	\item \textit{Notes on Diffy Qs} by Ji\v{r}\'{\i} Lebel, 2013/2014\\
	Available at \url{http://www.jirka.org/diffyqs/}
	\begin{itemize}
		\item Designed to work hand-in-hand with the IODE Software (more later)
		\item Also available VERY cheap in bound versions ($<$\$20)
	\end{itemize}
\end{itemize}
}

\frame{
\frametitle{Textbooks (cont)}
\begin{itemize}
	\item \textit{Differential Equations with Linear Algebra, An inquiry based approach to learning}, by Ben Woodruff\\
	Available from github: \tiny{\url{https://github.com/bmwoodruff/math316-IBL}}\normalsize
	\begin{itemize}
		\item Includes a nice review section and linear algebra section (for combined courses)
		\item Takes a constructivist/IBL approach to teaching with most key ideas `discovered' by the students
		\item Includes many links directly to Sage web-applets (or WolframAlpha) for answer checking
	\end{itemize}
	\item Unpublished IBL Project from Chris Rasmussen (contact him for materials)
\end{itemize}
}

\frame{
\frametitle{Project \& Assignment Sources}
\begin{itemize}
	\item First, the two textbooks (non-IBL) include several project/worksheets
	\item IODE, again at \url{www.math.uiuc.edu/iode/}
	\begin{itemize}
		\item Fairly stand-alone projects/labs - 6 total
		\item Inclues Fourier Series and PDEs
		\item Some small compatibility issues/updates
		\item Key/Solutions not posted, but I received some from an Admin/have available
	\end{itemize}
	\item The Connected Curriculum Project at \url{https://www.math.duke.edu/education/ccp}
	\begin{itemize}
		\item 14 Labs/projects (for DEs), in Maple, Mathematic, and MATLAB formats
		\item All are for OLD software versions, so try them first to revise some commands
		\item This site actually has a LOT of awesome (other) stuff
	\end{itemize}
\end{itemize}
}

\frame{
\frametitle{Project \& Assignment Sources 2}
\begin{itemize}
	\item CODEE at \url{www.codee.org}
	\begin{itemize}
		\item Most important:\\ The 2013 JMM Mini-Course workbook. \\
		\tiny{\url{http://www.codee.org/jmm-2013-minicourse/jmm-2013-project-book/view}}\normalsize
		\begin{itemize}
			\item This has a large breadth of projects of varying complexities and walk-through levels
			\item Biggest Downside: No clear list of pre-requisite knowledge (or sometimes learning goals) for projects
		\end{itemize}
		\item Very nice `Mathematical Modeling' process/diagram and activity
		\item Lots of other projects and resources if you dig around
	\end{itemize}
	\item MIT Open Courseware
	\begin{itemize}
		\item Includes some web-based applets (Javascript anyone?)
		\item Some problem-sets and solutions also available
	\end{itemize}
\end{itemize}
}

\section{Tools}

\frame{
\frametitle{(Free) Software}
\begin{itemize}
	\item Sage (Cloud) at \url{http://sagemath.org} 
	\begin{itemize}
		\item Available in the cloud or downloadable, free alternative to Mathematica, Maple (mostly), and MATLAB (also Octave)
	\end{itemize}
	\item WolframAlpha at \url{http://wolframalpha.com}
	\begin{itemize}
		\item Your students will use it. You should understand it.
		\item Advantage/Disadvantage: mostly a one-line entry option
		\item Don't forget additional applets or explorations that Wolfram has available
	\end{itemize}
	\item IODE at \url{www.math.uiuc.edu/iode/} 
	\begin{itemize}
		\item Package for MATLAB/Octave
		\item Includes several useful GUI exploration pieces
	\end{itemize}
\end{itemize}

}

\frame{
\frametitle{Assessment}
\begin{itemize}
	\item \textbf{S}tudent \textbf{A}ssessment of their \textbf{L}earning \textbf{G}ains (SALG)\\
	Found at \url{www.salgsite.org}
	\begin{itemize}
		\item Learner focused, with heavily validated assessment questions
		\item Recommended: SALG-M (Sage or Non-Sage) by Hassi \& Laursen\\
		A SALG instrument developed  and validated explicitly for IBL Mathematics courses
	\end{itemize}
	\item Paper: \textit{Students' Retention of Mathematical Knowledge and Skills in Differential Equations}, by Oh Nam Kwon, Karen Allen, and Chris Rasmussen, \textbf{School Science and Mathematics}, May 2005
	\begin{itemize}
		\item Contains a short list of assessment items related to developing a Concept Inventory for DEs
	\end{itemize}
	\item (Your idea here) -- No (to my knowledge) definitive concept inventory.
\end{itemize}
}

\section{Tips}

\frame{
\frametitle{General Tips}

\begin{itemize}
	\item Don't (dramatically) change approaches after the semester starts
	\begin{itemize}
		\item Students really appreciate predictability
	\end{itemize}
	\item If using IBL, go extra slow at the beginning
	\begin{itemize}
		\item without that foundation, the rest will fall apart
		\item Get support from the the Academy of Inquiry-Based Learning\\
		\url{www.inquirybasedlearning.org}
	\end{itemize}
\end{itemize}

}

\frame{
\frametitle{Reflections on Tools and Resources}
\begin{itemize}
	\item The IODE labs are well-written and reasonably easy to incorporate
	\begin{itemize}
		\item Students really enjoyed these, and were generally successful
		\item Some programming may need taught
	\end{itemize}
	\item The CODEE/JMM2013 Projects are great, but are not really ``plug and play''
	\begin{itemize}
		\item Projects need a clearer identification of what tools/skills they need to have mastered before the project or to complete the project
		\item Solutions are not always available, so may need solved by you first
		\item Some of these are very challenging and indepth!
	\end{itemize}
\end{itemize}
}

\section{Thanks}

\frame{
\Large{Big Thanks}\ \normalsize to Ben Woodruff for the textbook\\
\vskip0.1in
also\\
Chris Rasmussen for his draft text and assessment questions, \\
\vskip0.1in
and\\
Tom LoFaro, Michael Huber, and Ami Radunskaya for writing and sharing their modeling projects and many solutions via email.\\

}

\frame{
\Huge
Thank you for listening\\
\normalsize
Download slides at:\\
\url{https://github.com/Pelonza/JMM16-ODE-Talk}

}
		




\end{document}


%DNA picture from : http://www.pbs.org/wgbh/nova/sciencenow/3214/01-coll-04.html
%Molecule picture from: http://www.xray.cz/xray/csca/kol2009/abst/moncol.htm
% TSP: by Mac Evans, GTI, ResearchHorizons